\documentclass[a4paper]{article}
\usepackage[utf8]{inputenc}
\usepackage[english,russian]{babel}
\usepackage{a4wide}
\usepackage{enumitem}
\usepackage{multicol}
\usepackage[vmargin=1cm]{geometry}
\setenumerate{noitemsep}
\pagestyle{empty}


\begin{document}
\Large

\begin{center}
\hrule
\medskip

\textbf{\textsc{Søren Kierkegaard - Subjectivity, Irony and the Crisis of Modernity}}
\smallskip

{\large\textrm{Jon Stewart, PhD, Dr theol \& phil, Søren Kierkegaard research center, Kopenhagen University}}

\smallskip

\textsl{Осень 2013\,г. Недели 1--7.1}
\end{center}

\large

\centerline{\parbox{14cm}{\begin{description}
\item[Week 1:] Course Introduction: The Life and Work of Kierkegaard as a <<Socratic Task>>
\item[Week 2:] Kierkegaard, Martensen and Hegelianism at the University of Copenhagen
\item[Week 3:] Kierkegaard’s View of Socrates
\item[Week 4:] Kierkegaard, Heiberg and History
\item[Week 5:] Kierkegaard, P.M. Møller and Friedrich von Schlegel
\item[Week 6:] The Conception of Kierkegaard’s Socratic Task: 1843. The Trip to Berlin and the Beginning of the Authorship
\item[Week 7:] Kierkegaard’s Socratic Task: 1844-45. The Development of the Pseudonymous Works. Part 1.
\end{description}}}

\hrule

\newpage
\small
\begin{multicols}{2}
\begin{enumerate}
\item[\bf W1] \textsl{Introduction: The Life and Work of Kierkegaard as a <<Socratic Task>>}
\item  Course Introduction
\item  Kierkegaard and Socrates
\item  Kierkegaard’s Family and the Danish Golden Age
\item  Kierkegaard and the School of Civic Virtue
\item  Introduction to The Concept of Irony
\item  Socratic Irony and Ignorance
\item  Socrates and Aporia
\item  Socrates and the Sophists
\item  Socrates’ Mission and the Gadfly
\item The Daimon
\item Socrates’ Maieutics
\item Interview with Professor Peter Šajda, Slovak Academy of Science
\item Conclusion
\end{enumerate}

\begin{enumerate}
\item[\bf W2] \textsl{Kierkegaard, Martensen and Hegelianism at the University of Copenhagen}
\item  Introduction
\item  Hans Lassen Martensen and the Concept of Autonomy
\item  Martensen and the University of Copenhagen in the 1830s
\item  Introduction to Hegel
\item  Hegel’s Understanding of Socrates in the History of Philosophy
\item  Sophocles’ Antigone
\item  A Truth for Which to Live and Die
\item  Hegel’s View of the Socratic Method and Irony
\item  Hegel’s Interpretation of Maieutics
\item  Hegel’s Interpretation of Aporia
\item  Hegel’s Interpretation of Socrates and the Good
\item  Hegel’s Interpretation of Socrates and the Sophists
\item  Hegel’s Interpretation of the Daimon of Socrates
\item  Hegel’s Analysis of Socrates’ Trial, Part 1: The Charges
\item  Hegel’s Analysis of Socrates’ Trial, Part 2: The Sentence
\item  Conclusion
\end{enumerate}

\begin{enumerate}
\item[\bf W3] \textsl{Kierkegaard’s View of Socrates}
\item  Introduction
\item  Kierkegaard’s View of the Daimon of Socrates
\item  Martensen’s Faust
\item  Kierkegaard’s Analysis of Socrates’ Trial, Part 1: The Charges
\item  Kierkegaard’s Analysis of Socrates’ Trial, Part 2: The Sentence
\item  Doubt and The Conflict between the Old and the New Soap Cellars
\item  Kierkegaard’s Johannes Climacus or De Omnibus dubitandum est
\item  The View Made Necessary: The Sophists
\item  Socrates and Christ
\item  Andreas Frederik Beck and the First Review of The Concept of Irony
\item  Interview with Brian Soderquist
\item  The Legacy of Socrates
\item  Conclusion
\end{enumerate}

\begin{enumerate}
\item[\bf W4] \textsl{Kierkegaard, Heiberg and History}
\item  Introduction
\item  Kierkegaard’s Introduction to Part Two
\item  German Romanticism
\item  Kierkegaard’s <<Observations for Orientation>>
\item  Johan Ludvig Heiberg’s On the Significance of Philosophy for the Present Age
\item  Kierkegaard’s <<The World Historical Validity of Irony>>
\item  Hegel on Socratic and Romantic Irony
\item  Kierkegaard’s Criticism of Hegel’s View of Irony
\item  Kierkegaard’s First Criticism of Hegel: The Relation of Philosophy to History
\item  Interview with Heiko Schulz, Johann Wolfgang Goethe University
\item  Kierkegaard’s Second Criticism of Hegel: Socrates as a Negative Figure
\item  Conclusion
\end{enumerate}

\begin{enumerate}
\item[\bf W5] \textsl{Kierkegaard, P.M. Møller and Friedrich von Schlegel}
\item  Introduction
\item  Introduction to Fichte
\item  Hegel’s Analysis of Fichte
\item  Kierkegaard’s Analysis of Fichte
\item  The Appropriation of Fichte’s Theory by Schlegel and Tieck
\item  Kierkegaard’s Analysis of Schlegel
\item  Kierkegaard and Poul Martin Møller
\item  Interview with Brian Soderquist
\item  Kierkegaard’s Idea of Controlled Irony
\item  Kierkegaard’s Defense and the Reception of the Work
\item  Interview with Karl Verstrynge
\item  Kierkegaard and Regine Olsen
\item  The Broken Engagement
\item  Conclusion
\end{enumerate}

\begin{enumerate}
\item[\bf W6] \textsl{The Conception of Kierkegaard’s Socratic Task: 1843. The Trip to Berlin and the Beginning of the Authorship}
\item Introduction
\item Кьеркегор в Берлине
\item Споры о среднем и концепция <<Или-или>>
\item Диапсалмата. Эстет, как иронизирующий романтик
\item Прием <<Или-или>> критикой
\item Последующие работы Кьеркегора
\item Всеобщее и отдельный индивид
\item Парадокс веры
\item Interview with Heiko Schulz, Johann Wolfgang Goethe University
\item Conclusion
\end{enumerate}

\begin{enumerate}
\item[\bf W7] \textsl{Kierkegaard’s Socratic Task: 1844-45. The Development of the Pseudonymous Works}
\item Introduction
\item <<Философские крохи>>
\end{enumerate}
\end{multicols}
\newpage

\hrule
\begin{enumerate}
\item[\bf W7] \textsl{Kierkegaard’s Socratic Task: 1844-45. The Development of the Pseudonymous Works}
\setcounter{enumi}{2}
\item <<Понятие страха>> (Concept of Anxiety)
\item <<Предисловия>> и полемика с Гейбергом
\item <<Предисловие VIII>>
\item <<Этапы жизненного пути>>
\item Конфликт с журналом Corsair
\item Введение в <<Заключительное ненаучное послесловие>>
\item <<Первое и последнее разъяснение>>
\item Parallel Authorship
\item Дневники и тетради Кьеркегора
\item Conclusion
\end{enumerate}

\begin{enumerate}
\item[\bf W8] \textsl{The 2nd Half of the Authorship and the Attack on the Church (8.1, 8.2)}
\item Introduction
\item Взгляд Кьеркегора на общество. Его отношения с королем Кристианом VIII
\item Работы после <<Заключительного ненаучного послесловия>>
\item <<The Point of View>> Кьеркегора
\item Революция 1848 г. и <<Болезнь к смерти>>
\item <<Practice in Christianity>>
\item Нападки Кьеркегора на церковь Дании
\end{enumerate}
\hrule
\newpage

\hrule
\begin{enumerate}
\item[\bf W8] \textsl{The 2nd Half of the Authorship and the Attack on the Church (8.3, 8.4)}
\setcounter{enumi}{7}
\item Последний выпуск <<The Moment>>
\item Болезнь и смерть Кьеркегора
\item Похороны Кьеркегора
\item Интервью с проф. Daniel Conway, USA
\item Интервью с проф. Wang Qi, КНР
\item Наследие Кьеркегора
\item Conclusion
\end{enumerate}
\hrule

\end{document}
